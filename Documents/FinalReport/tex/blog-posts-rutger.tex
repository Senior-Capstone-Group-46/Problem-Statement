\subsection{Rutger Farry}
\subsubsection{October 14, 2016}\label{section}
This week was focused on battening down the hatches on what our final
product is going to look like. Up until last Friday, I only knew the
platform and language our program would be implemented in. I had no idea
what it would actually do, what it should look like, etc.

I now understand we will be producing a fitness app for iOS that
connects to the Mindbody platform to communicate workouts between
clients and coaches. We narrowed down the problem we'd be addressing in
our
\href{https://github.com/iOS-Capstone/Problem-Statement/blob/master/ProblemStatement/ProblemStatement.pdf}{Problem
Statement}, drew up wireframes, and began investigating what libraries
we should use for things such as API calls, layouts, data management,
etc.

Next week we will be driving up to Portland to meet with our team in the
eBay offices, and should be able to begin work on the app. I'd like to
get the structure outlined completely in code, and perhaps begin on some
API calls. Looking forward to see what we can get done!

\subsubsection{October 21, 2016}\label{section}
The highlight of this week was traveling to Portland and visiting the
people at eBay's office that we'd be working with. We had a good
high-level talk about the app's purpose and later drilled down into the
specific requirements, with a more extensive functional requirements doc
promised soon.

I now feel like we're nearing the level of confidence in the app's
direction that we'll be able to start coding soon. The next steps we
take are mostly to improve the quality of the code we write. Things like
setting up continuous integration and developing UI redlines to create
consistent expectations across team members and to ensure that those
expectations are met. We will probably be focusing on this, along with
writing / revising the class's homework documents before moving on to
code.

\subsubsection{October 28, 2016}\label{section}
We're getting into the thick of midterm season, which I think has
everyone a little stretched. Most of our efforts this week were focused
on getting our Requirements Document written and our Problem Statement
revised. Working on the requirements document actually encouraged us to
get some useful work done on designing our application architecture,
however. I'm excited that we finally started drilling down into this and
am excited to explore some of the decisions we made in the area of
functional reactive programming and unidirectional dataflow in Swift.

The team has additionally been looking into where we want the UI design
to go, but it hasn't progressed much beyond high-level discussions and
looking at design inspirations. Hopefully as midterms start to wind down
we can start laying down the foundation of the app.

\subsubsection{November 4, 2016}\label{section}
Progress is coming along well on the design and I feel that we have a
very solid foundation to move forward on. We've written several design
documents planning out the different stages of our implementation and
have considered how to construct the different components of our app and
how they will interact. With midterms winding down for me, I'm looking
forward to starting to implement the app.

\subsubsection{November 11, 2016}\label{section}
We've started work on another document -- a technology review to further
explore possible implementation plans for the components that make up
our app and to choose the best option. In my opinion, the best way to do
this is to just start writing something with that library /
architecture. The time constraints for the document make it impossible
to do this effectively however, so we are just reading blog posts and
software documentation from Apple instead.

Both us and our sponsors are excited for us to begin implementing our
app, but most our work has gone into writing our documents however.

\subsubsection{November 18, 2016}\label{section}
This week was mostly coasting. We worked on our technical requirements
doc and have started looking into the design doc. We've had a bit of
communication with eBay about our mockups, but besides that haven't
talked with them much. I've been working on some side iOS projects and
have a good idea about how I'd like to implement state management in our
app.

\subsubsection{November 25, 2016}\label{section}
This week was mostly spent eating turkey. I also ate some other things.
Didn't do much related to the app.

\subsubsection{December 2, 2016}\label{section}
This week we worked on our design document along with the myriad of
other homeworks we had assigned. We ended up writing quite a lot and
turned it in Friday morning. We're looking forward to using the lack of
homework over Christmas break to start working on the app.

\subsubsection{January 13, 2017}\label{section}
Winter break is over, and surprise(!) we didn't start working on the app. 
At least I can say that we are refreshed and ready to begin. We are 
planning to meet up with eBay soon to refine our designs before starting 
implementation in earnest.

\subsubsection{January 20, 2017}\label{section}
We met with eBay over a weird Skype clone. Something came up over the 
winter and it turns out the MINDBODY API costs like \$25k a year (which 
is surprising since it's trash), so we are pivoting from using that to 
just making our stretch goal of implementing a fitness store using the 
eBay API into a primary goal. This shouldn't be too hard since we've 
already done some planning in this area, but should probably flesh 
those plans out a bit more it is now a primary goal.

\subsubsection{January 27, 2017}\label{section}
We still have not done much with implementation and now midterms are 
starting. We are feeling an initial sense of urgency and have decided 
to start holding weekly meetings to code.

\subsubsection{February 3, 2017}\label{section}
The alpha build is due soon. We've started coding in earnest and split 
out work into issues on GitHub. We've basically just made branches for 
every main screen and assigned everyone a branch.

\subsubsection{February 10, 2017}\label{section}
Worked on laying out the homescreen using a UICollectionView. This was 
pretty nice and allowed for a good amount of flexibility that I think 
we might want if we ever make this into an iPad app.

\subsubsection{February 17, 2017}\label{section}
We merged our branches this week and have an alpha build and midterm 
report due >.< We are going to reassess what remaining work needs to 
be done and will make some new issues and branches to split this work 
up.

\subsubsection{February 24, 2017}\label{section}
I'm kinda sad that we've gotten Firebase so deeply integrated into 
our app. While it is convenient, I've heard horror stories of companies 
who've built a similar reliance on a third-party service deeply into 
their application and then suffered when the service was shuttered 
or increased their prices.

In other news, we demoed our app to our TA today and it seemed to go 
well.

\subsubsection{March 3, 2017}\label{section}
Pretty much my only focus right now is organizing the 2017 Randall 
Fox Beaver Omnium this weekend and making sure that goes smoothly. 
I will be on the race course all day and probably only sleep about 
4 hours this weekend, so am not going to work on the app.

\subsubsection{March 10, 2017}\label{section}
Still haven't worked on the app a lot. Need to start work on our term 
report and videos, as well as making a rough draft of our Expo poster.
Cycling season has begun though, and although I'm not working to organize
a race anymore, I will still be traveling / racing every weekend for the
next six weeks, so won't have a lot of free time.

\subsubsection{March 17, 2017}\label{section}
I've been able to work on the app a bit during the week, but weekends are 
totally hectic. Luckily my course load is pretty easy and I'm not to 
pressured to work on school / the app during the weekend, but time during 
the week is definitely tight.

\subsubsection{March 24, 2017}\label{section}
Had a very hectic time submitting my group's term presentation in Seattle
this weekend (we are racing at UW). Upon arriving at our home stay at 10pm,
I holed myself up in our truck and recorded my portion of the app demo and 
then hitched a ride to a Starbucks so I could upload the 700MB file on 
my laptop. All next week I will be skiing around Montana with probably very
rare access to WiFi, so will not be working on the app at all.

\subsubsection{April 7, 2017}\label{section}
After some fun skiing in Montana, our team came back from spring break
refreshed and ready to work. I think we actually hit the ground
running pretty well; we merged our progress together into \texttt{dev},
with Brandon and I merging first. (Michael finished his branch a few
days later and had to resolve a bunch of merge conflicts). We met
several times to resolve these as a team and got everything merged up by
Thursday.

\subsubsection{April 14, 2017}\label{section}
Last weekend I got sick and got a really nasty ear infection on my way
to my bike race at Whitman. Sat out for most of the events, but raced
in the criterium. Upon returning on Sunday, I got so sick that I slept
all day Monday and have been in pain all week. Didn't touch C7Fit.

\subsubsection{April 21, 2017}\label{section}
Still have an ear infection and have been taking it easy. I did some
work on adding the trainer cell to the home screen. Now tapping on the
trainer cell will open a list of trainers, but the list is kinda ugly
and stretches images weird. Will probably work on that next week.

\subsubsection{April 28, 2017}\label{section}
Ear infection is abating and I'm starting to feel normal again. Worked a
lot on a React Native side project during my illness but not too much on
the app. This week I finished up the home screen for the most part
though and merged my UI branch that cleans up the colors in the navbar
and other areas.

\subsubsection{May 5, 2017}\label{section}
This week was super crazy. We went up to Portland on Monday to meet with
eBay, which we didn't really have time for, but knew it needed to be
done. (Did a little polishing on the app the night before, too). This
week I had \textbf{nearly a million assignments due:} – 2 assignments in
Operating Systems II – 2 assignments in AI – 2 assignments in Computer
Architecture – 1 midterm in Computer Architecture – and had to work on
the WIRED article for this class

Needless to say, no work was done on C7FIT.

\subsubsection{May 12, 2017}\label{section}
This week was a lot more chill, but still had a bit to do. Made
everything about the app was pretty and ready for our midterm report /
expo. Made the midterm report and video.

\subsubsection{May 19, 2017}\label{section}
\paragraph{If you were to redo the project from Fall term, what would
you tell
yourself?}\label{if-you-were-to-redo-the-project-from-fall-term-what-would-you-tell-yourself}

I would tell myself to do more work at the beginning to establish coding
idioms for the project. I think that since we all started at about the
same time, no one really looked at each other's work and for that reason
our styles are all a little different. I'd also enable our linter and
tests before the project started to ensure everyone is following a
consistent style from the start. Changing our code to conform to our
style guidelines halfway through the project was a little painful.

I'd also try to maintain more open communication with our client. They
were pretty hands-off, but I think if we made the effort to communicate
more we might have made the development process smoother for ourselves.

Finally I'd tell everyone on the team to commit / make PRs more. We
worked on big feature branches that would consume a whole screen, and
while siloing ourselves to screens was good, we still had some nasty
merges and probably would've had more shared code if we merged more.

\paragraph{What's the biggest skill you've
learned?}\label{whats-the-biggest-skill-youve-learned}

Better discipline is probably the biggest takeaway from this. There were
some times when there was really no pressure to work on the project.
Well I think we did a decent job pacing ourselves, we still could've
done better and avoided some mad dashes.

\paragraph{What skills do you see yourself using in the
future?}\label{what-skills-do-you-see-yourself-using-in-the-future}

Discipline / task management.

\paragraph{What did you like about the project, and what did you
not?}\label{what-did-you-like-about-the-project-and-what-did-you-not}

The thing I liked about the project was also the thing I disliked. It
was easy. Making an app is not breakthrough and is something I've lots
of times before. While I enjoyed the extra free time to work on my own
projects and go on bike races, part of me wishes that I worked on a
project that captured my interest more, even if it would have taken more
time.

\paragraph{What did you learn from your
teammates?}\label{what-did-you-learn-from-your-teammates}

Brandon was super good at staying on top of things and scheduling out
his work. I'd like to be more like that on the next project I do.
Michael got up to speed on iOS super quickly and was a very good Swift
developer by the end of the project. I don't know how he did it, but I
should probably ask.

\paragraph{If you were the client for this project, would you be
satisfied with the work
done?}\label{if-you-were-the-client-for-this-project-would-you-be-satisfied-with-the-work-done}

Yes. App development is crazy expensive. Getting an app of similar
quality developed by an US-based iOS shop would cost at least \$30,000,
and very likely much more. That being said, I think the app is overall
very responsive and with a few tweaks, the user interface would be
top-tier. Additionally, very few, if any gyms at the scale of
\href{http://www.clubsevenfitness.com/}{Club Seven Fitness} have an iOS
app, or even a quality website. An iOS app will hopefully help them
market their gym and provide a useful utility to current club members.

\paragraph{If your project were to be continued next year, what do you
think needs to be working
on?}\label{if-your-project-were-to-be-continued-next-year-what-do-you-think-needs-to-be-working-on}

We'd probably start by working on feedback from our App Store launch.
This would probably be mostly UI tweaks. Afterward, I think it'd be
useful to build our own scheduling API and integrate that into the app,
since the \href{https://www.mindbodyonline.com/}{MINDBODY} API is
terrible \emph{and} extremely expensive. Finally, building a web
frontend for coaches to more easily view their client's workouts would
be a cool expansion of the project, since using Firebase might be a
little clunky for non-developers.
